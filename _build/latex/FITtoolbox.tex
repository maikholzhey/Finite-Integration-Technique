%% Generated by Sphinx.
\def\sphinxdocclass{report}
\documentclass[letterpaper,10pt,english]{sphinxmanual}
\ifdefined\pdfpxdimen
   \let\sphinxpxdimen\pdfpxdimen\else\newdimen\sphinxpxdimen
\fi \sphinxpxdimen=.75bp\relax

\PassOptionsToPackage{warn}{textcomp}
\usepackage[utf8]{inputenc}
\ifdefined\DeclareUnicodeCharacter
% support both utf8 and utf8x syntaxes
\edef\sphinxdqmaybe{\ifdefined\DeclareUnicodeCharacterAsOptional\string"\fi}
  \DeclareUnicodeCharacter{\sphinxdqmaybe00A0}{\nobreakspace}
  \DeclareUnicodeCharacter{\sphinxdqmaybe2500}{\sphinxunichar{2500}}
  \DeclareUnicodeCharacter{\sphinxdqmaybe2502}{\sphinxunichar{2502}}
  \DeclareUnicodeCharacter{\sphinxdqmaybe2514}{\sphinxunichar{2514}}
  \DeclareUnicodeCharacter{\sphinxdqmaybe251C}{\sphinxunichar{251C}}
  \DeclareUnicodeCharacter{\sphinxdqmaybe2572}{\textbackslash}
\fi
\usepackage{cmap}
\usepackage[T1]{fontenc}
\usepackage{amsmath,amssymb,amstext}
\usepackage{babel}
\usepackage{times}
\usepackage[Bjarne]{fncychap}
\usepackage{sphinx}

\fvset{fontsize=\small}
\usepackage{geometry}

% Include hyperref last.
\usepackage{hyperref}
% Fix anchor placement for figures with captions.
\usepackage{hypcap}% it must be loaded after hyperref.
% Set up styles of URL: it should be placed after hyperref.
\urlstyle{same}
\addto\captionsenglish{\renewcommand{\contentsname}{Contents:}}

\addto\captionsenglish{\renewcommand{\figurename}{Fig.}}
\addto\captionsenglish{\renewcommand{\tablename}{Table}}
\addto\captionsenglish{\renewcommand{\literalblockname}{Listing}}

\addto\captionsenglish{\renewcommand{\literalblockcontinuedname}{continued from previous page}}
\addto\captionsenglish{\renewcommand{\literalblockcontinuesname}{continues on next page}}
\addto\captionsenglish{\renewcommand{\sphinxnonalphabeticalgroupname}{Non-alphabetical}}
\addto\captionsenglish{\renewcommand{\sphinxsymbolsname}{Symbols}}
\addto\captionsenglish{\renewcommand{\sphinxnumbersname}{Numbers}}

\addto\extrasenglish{\def\pageautorefname{page}}

\setcounter{tocdepth}{3}
\setcounter{secnumdepth}{3}


\title{FITtoolbox Documentation}
\date{Apr 08, 2019}
\release{1.2}
\author{maik.holzhey}
\newcommand{\sphinxlogo}{\vbox{}}
\renewcommand{\releasename}{Release}
\makeindex
\begin{document}

\pagestyle{empty}
\maketitle
\pagestyle{plain}
\sphinxtableofcontents
\pagestyle{normal}
\phantomsection\label{\detokenize{index::doc}}



\chapter{FIT module}
\label{\detokenize{FIT:module-FIT}}\label{\detokenize{FIT:fit-module}}\label{\detokenize{FIT::doc}}\index{FIT (module)@\spxentry{FIT}\spxextra{module}}
This module is a generic PYTHON implementation of the essential functions for doing computations using the Finite-Integration-Technique

\begin{sphinxadmonition}{warning}{Warning:}
1/2 = 0 !integer division!
\end{sphinxadmonition}
\begin{description}
\item[{Use instead:}] \leavevmode
\fvset{hllines={, ,}}%
\begin{sphinxVerbatim}[commandchars=\\\{\}]
\PYG{g+gp}{\PYGZgt{}\PYGZgt{}\PYGZgt{} }\PYG{l+m+mf}{1.}\PYG{o}{/}\PYG{l+m+mi}{2} \PYG{o}{=} \PYG{l+m+mf}{0.5}
\end{sphinxVerbatim}

\item[{This module can be imported using}] \leavevmode
\fvset{hllines={, ,}}%
\begin{sphinxVerbatim}[commandchars=\\\{\}]
\PYG{g+gp}{\PYGZgt{}\PYGZgt{}\PYGZgt{} }\PYG{k+kn}{import} \PYG{n+nn}{FIT}
\end{sphinxVerbatim}

\end{description}

The main function calculating an example does only execute when module runs in main frame of the execution stack, which is not the case for \sphinxtitleref{import} statements.
\index{FIT (class in FIT)@\spxentry{FIT}\spxextra{class in FIT}}

\begin{fulllineitems}
\phantomsection\label{\detokenize{FIT:FIT.FIT}}\pysigline{\sphinxbfcode{\sphinxupquote{class }}\sphinxcode{\sphinxupquote{FIT.}}\sphinxbfcode{\sphinxupquote{FIT}}}
This class-object represents the basic framework for any FIT calculations
\index{boundaryIndices() (FIT.FIT method)@\spxentry{boundaryIndices()}\spxextra{FIT.FIT method}}

\begin{fulllineitems}
\phantomsection\label{\detokenize{FIT:FIT.FIT.boundaryIndices}}\pysiglinewithargsret{\sphinxbfcode{\sphinxupquote{boundaryIndices}}}{\emph{boundaryConditions}}{}~\begin{description}
\item[{Boundary Indices:}] \leavevmode
determine the indices of boundary components (transversal e and normal h)
boundaryConditions = array of coded boundary info {[}xlow xhigh ylow yhigh zlow zhigh{]} 
1 = boundaries to be considered (i.e. PEC-bounds)

\end{description}

PMC (Neumann) Boundary conditions are given by default with the FIT operator stencil at boundary. So only if PEC = 1 this function makes an alteration to the given operators

Initializes attributes
\begin{quote}\begin{description}
\item[{Parameters}] \leavevmode
\sphinxstyleliteralstrong{\sphinxupquote{boundaryConditions}} (\sphinxstyleliteralemphasis{\sphinxupquote{integer array}}) \textendash{} array referencing {[}xlow xhigh ylow yhigh zlow zhigh{]}

\item[{Variables}] \leavevmode\begin{itemize}
\item {} 
\sphinxstyleliteralstrong{\sphinxupquote{epsBOUND}} \textendash{} array of indices of e-components

\item {} 
\sphinxstyleliteralstrong{\sphinxupquote{mueBOUND}} \textendash{} array of indices of h-components

\end{itemize}

\end{description}\end{quote}

\end{fulllineitems}

\index{boundaryIndicesBOR() (FIT.FIT method)@\spxentry{boundaryIndicesBOR()}\spxextra{FIT.FIT method}}

\begin{fulllineitems}
\phantomsection\label{\detokenize{FIT:FIT.FIT.boundaryIndicesBOR}}\pysiglinewithargsret{\sphinxbfcode{\sphinxupquote{boundaryIndicesBOR}}}{\emph{boundaryConditions}}{}~\begin{description}
\item[{Boundary Indices:}] \leavevmode
same as ‘boundaryIndices’ but for rotational body (rz-code)

\end{description}

\begin{sphinxadmonition}{warning}{Warning:}
If called, will raise NotImplementedError
\end{sphinxadmonition}

\end{fulllineitems}

\index{buildMaterial() (FIT.FIT method)@\spxentry{buildMaterial()}\spxextra{FIT.FIT method}}

\begin{fulllineitems}
\phantomsection\label{\detokenize{FIT:FIT.FIT.buildMaterial}}\pysiglinewithargsret{\sphinxbfcode{\sphinxupquote{buildMaterial}}}{}{}
Builds material matrices and stores them as attributes:

Also processes boundary information to represent the model geometry
\begin{quote}\begin{description}
\item[{Variables}] \leavevmode\begin{itemize}
\item {} 
\sphinxstyleliteralstrong{\sphinxupquote{Meps}} \textendash{} FIT epsilon matrix

\item {} 
\sphinxstyleliteralstrong{\sphinxupquote{Mepsi}} \textendash{} FIT epsilon matrix inverse

\item {} 
\sphinxstyleliteralstrong{\sphinxupquote{Mmue}} \textendash{} FIT mue matrix

\item {} 
\sphinxstyleliteralstrong{\sphinxupquote{Mmuei}} \textendash{} FIT mue matrix inverse

\end{itemize}

\end{description}\end{quote}

\end{fulllineitems}

\index{createOperators() (FIT.FIT method)@\spxentry{createOperators()}\spxextra{FIT.FIT method}}

\begin{fulllineitems}
\phantomsection\label{\detokenize{FIT:FIT.FIT.createOperators}}\pysiglinewithargsret{\sphinxbfcode{\sphinxupquote{createOperators}}}{}{}
FIT operators

Intializes the attributes:
\begin{quote}\begin{description}
\item[{Variables}] \leavevmode\begin{itemize}
\item {} 
\sphinxstyleliteralstrong{\sphinxupquote{G}} \textendash{} gradient operator

\item {} 
\sphinxstyleliteralstrong{\sphinxupquote{S}} \textendash{} divergence operator

\item {} 
\sphinxstyleliteralstrong{\sphinxupquote{C}} \textendash{} curl operator

\item {} 
\sphinxstyleliteralstrong{\sphinxupquote{Cs}} \textendash{} curl operator on dual grid

\end{itemize}

\end{description}\end{quote}

\end{fulllineitems}

\index{createOperatorsBOR() (FIT.FIT method)@\spxentry{createOperatorsBOR()}\spxextra{FIT.FIT method}}

\begin{fulllineitems}
\phantomsection\label{\detokenize{FIT:FIT.FIT.createOperatorsBOR}}\pysiglinewithargsret{\sphinxbfcode{\sphinxupquote{createOperatorsBOR}}}{}{}
FIT operators: rz code

\begin{sphinxadmonition}{warning}{Warning:}
If called, will raise NotImplementedError
\end{sphinxadmonition}

\end{fulllineitems}

\index{createP() (FIT.FIT method)@\spxentry{createP()}\spxextra{FIT.FIT method}}

\begin{fulllineitems}
\phantomsection\label{\detokenize{FIT:FIT.FIT.createP}}\pysiglinewithargsret{\sphinxbfcode{\sphinxupquote{createP}}}{}{}
FIT elementary operators

Initializes the attributes:
\begin{quote}\begin{description}
\item[{Variables}] \leavevmode\begin{itemize}
\item {} 
\sphinxstyleliteralstrong{\sphinxupquote{Px}} \textendash{} Px as FIT operator

\item {} 
\sphinxstyleliteralstrong{\sphinxupquote{Py}} \textendash{} Py as FIT operator

\item {} 
\sphinxstyleliteralstrong{\sphinxupquote{Pz}} \textendash{} Pz as FIT operator

\end{itemize}

\end{description}\end{quote}

\end{fulllineitems}

\index{createPBOR() (FIT.FIT method)@\spxentry{createPBOR()}\spxextra{FIT.FIT method}}

\begin{fulllineitems}
\phantomsection\label{\detokenize{FIT:FIT.FIT.createPBOR}}\pysiglinewithargsret{\sphinxbfcode{\sphinxupquote{createPBOR}}}{}{}
FIT elementary operators: rz code

\begin{sphinxadmonition}{warning}{Warning:}
If called, will raise NotImplementedError
\end{sphinxadmonition}

\end{fulllineitems}

\index{defaultModel() (FIT.FIT method)@\spxentry{defaultModel()}\spxextra{FIT.FIT method}}

\begin{fulllineitems}
\phantomsection\label{\detokenize{FIT:FIT.FIT.defaultModel}}\pysiglinewithargsret{\sphinxbfcode{\sphinxupquote{defaultModel}}}{\emph{nx}, \emph{ny}, \emph{nz}, \emph{LxMin}, \emph{LyMin}, \emph{LzMin}, \emph{LxMax}, \emph{LyMax}, \emph{LzMax}, \emph{boundaryConditions}}{}
A wrapper function to have everything as quickly as possible
\begin{quote}\begin{description}
\item[{Parameters}] \leavevmode\begin{itemize}
\item {} 
\sphinxstyleliteralstrong{\sphinxupquote{N}} (\sphinxstyleliteralemphasis{\sphinxupquote{integer}}) \textendash{} number of points in \_

\item {} 
\sphinxstyleliteralstrong{\sphinxupquote{L\_Min}} (\sphinxstyleliteralemphasis{\sphinxupquote{float}}) \textendash{} total length in \_ (minimum)

\item {} 
\sphinxstyleliteralstrong{\sphinxupquote{L\_Max}} (\sphinxstyleliteralemphasis{\sphinxupquote{float}}) \textendash{} total length in \_ (maximum)

\end{itemize}

\end{description}\end{quote}
\begin{description}
\item[{Boundary Indices:}] \leavevmode
determine the indices of boundary components (transversal e and normal h)
boundaryConditions = array of coded boundary info {[}xlow xhigh ylow yhigh zlow zhigh{]} 
1 = boundaries to be considered (i.e. PEC-bounds)

\end{description}

PML (Neumann) Boundary conditions are given by default with the FIT operator stencil at boundary. So only if PEC = 1 this function makes an alteration to the given operators
\begin{quote}\begin{description}
\item[{Parameters}] \leavevmode
\sphinxstyleliteralstrong{\sphinxupquote{boundaryConditions}} (\sphinxstyleliteralemphasis{\sphinxupquote{integer array}}) \textendash{} array referencing {[}xlow xhigh ylow yhigh zlow zhigh{]}

\end{description}\end{quote}

Initializes attributes
\begin{quote}\begin{description}
\item[{Variables}] \leavevmode\begin{itemize}
\item {} 
\sphinxstyleliteralstrong{\sphinxupquote{epsBOUND}} \textendash{} array of indices of e-components

\item {} 
\sphinxstyleliteralstrong{\sphinxupquote{mueBOUND}} \textendash{} array of indices of h-components

\end{itemize}

\end{description}\end{quote}

All of the above are then accessible via attribute calling convention
\begin{description}
\item[{Example usage}] \leavevmode
\fvset{hllines={, ,}}%
\begin{sphinxVerbatim}[commandchars=\\\{\}]
\PYG{g+gp}{\PYGZgt{}\PYGZgt{}\PYGZgt{} }\PYG{n}{fitObject}\PYG{o}{.}\PYG{n}{nx}
\end{sphinxVerbatim}

\end{description}

\end{fulllineitems}

\index{defaultModelBOR() (FIT.FIT method)@\spxentry{defaultModelBOR()}\spxextra{FIT.FIT method}}

\begin{fulllineitems}
\phantomsection\label{\detokenize{FIT:FIT.FIT.defaultModelBOR}}\pysiglinewithargsret{\sphinxbfcode{\sphinxupquote{defaultModelBOR}}}{}{}
A wrapper function to have everything as quickly as possible: rz code

\begin{sphinxadmonition}{warning}{Warning:}
If called, will raise NotImplementedError
\end{sphinxadmonition}

\end{fulllineitems}

\index{dftMatrix() (FIT.FIT method)@\spxentry{dftMatrix()}\spxextra{FIT.FIT method}}

\begin{fulllineitems}
\phantomsection\label{\detokenize{FIT:FIT.FIT.dftMatrix}}\pysiglinewithargsret{\sphinxbfcode{\sphinxupquote{dftMatrix}}}{\emph{frequency}, \emph{timeSignalLength}, \emph{dtmax}}{}
Constructs a linear operator, the DFT matrix, with a frequency selective sampling for use in \sphinxtitleref{selectiveDFT()}
\begin{quote}\begin{description}
\item[{Parameters}] \leavevmode\begin{itemize}
\item {} 
\sphinxstyleliteralstrong{\sphinxupquote{frequency}} (\sphinxstyleliteralemphasis{\sphinxupquote{numpy array}}) \textendash{} array with frequencies to be sampled

\item {} 
\sphinxstyleliteralstrong{\sphinxupquote{timeSignalLength}} (\sphinxstyleliteralemphasis{\sphinxupquote{number}}) \textendash{} length of timeSignal to norm result

\item {} 
\sphinxstyleliteralstrong{\sphinxupquote{dtmax}} \textendash{} time step (sampling)

\end{itemize}

\item[{Returns}] \leavevmode
dft matrix

\end{description}\end{quote}

\end{fulllineitems}

\index{fft() (FIT.FIT method)@\spxentry{fft()}\spxextra{FIT.FIT method}}

\begin{fulllineitems}
\phantomsection\label{\detokenize{FIT:FIT.FIT.fft}}\pysiglinewithargsret{\sphinxbfcode{\sphinxupquote{fft}}}{\emph{dtmax}, \emph{timeSignal}}{}
Normal Fast Fourier Transform for time signals.

Use for long time signals only, otherwise spectrum does not look very nice or use zero padding to increase frequency resolution. It is the fastest way to compute a frequency curve for an application. It uses a blackman window to mitigate wrong frequency  modulation due to finite length time signal.
\begin{quote}\begin{description}
\item[{Parameters}] \leavevmode\begin{itemize}
\item {} 
\sphinxstyleliteralstrong{\sphinxupquote{dtmax}} (\sphinxstyleliteralemphasis{\sphinxupquote{number}}) \textendash{} time step (“sampling”)

\item {} 
\sphinxstyleliteralstrong{\sphinxupquote{timeSignal}} (\sphinxstyleliteralemphasis{\sphinxupquote{numpy array}}) \textendash{} time signal to be analysed

\end{itemize}

\end{description}\end{quote}

\end{fulllineitems}

\index{fieldEnergy() (FIT.FIT method)@\spxentry{fieldEnergy()}\spxextra{FIT.FIT method}}

\begin{fulllineitems}
\phantomsection\label{\detokenize{FIT:FIT.FIT.fieldEnergy}}\pysiglinewithargsret{\sphinxbfcode{\sphinxupquote{fieldEnergy}}}{\emph{em1}, \emph{e}, \emph{h}}{}
Calculates the field energy of the discrete system
\begin{quote}\begin{description}
\item[{Parameters}] \leavevmode\begin{itemize}
\item {} 
\sphinxstyleliteralstrong{\sphinxupquote{e}}\sphinxstyleliteralstrong{\sphinxupquote{,}}\sphinxstyleliteralstrong{\sphinxupquote{h}} (\sphinxstyleliteralemphasis{\sphinxupquote{sparse array csc\_matrix}}) \textendash{} e,h field column vector in sparse (FIT algebraic dimension)

\item {} 
\sphinxstyleliteralstrong{\sphinxupquote{em1}} (\sphinxstyleliteralemphasis{\sphinxupquote{sparse array csc\_matrix}}) \textendash{} leapfrog update n-1 e field vector (FIT algebraic dimension)

\end{itemize}

\item[{Returns}] \leavevmode
scalar representing the momentary field energy

\end{description}\end{quote}

\end{fulllineitems}

\index{fieldMonitor() (FIT.FIT method)@\spxentry{fieldMonitor()}\spxextra{FIT.FIT method}}

\begin{fulllineitems}
\phantomsection\label{\detokenize{FIT:FIT.FIT.fieldMonitor}}\pysiglinewithargsret{\sphinxbfcode{\sphinxupquote{fieldMonitor}}}{\emph{data}, \emph{dataMonitor}, \emph{frequency}, \emph{lenTimeSignal}, \emph{stepIndex}, \emph{dtmax}}{}
Field Monitor to assess field solution at given frequency point. Needs to be placed within for-loop of time domain simulation. Essentially performs DFT at a given frequency point. Take care to run as many time steps so that the monitored frequency can be resolved.

dataMonitor is returned and needs to be stored. If it is an attribute of a FIT object this function would not automatically modify the content of the dataMonitor container
\begin{quote}\begin{description}
\item[{Parameters}] \leavevmode\begin{itemize}
\item {} 
\sphinxstyleliteralstrong{\sphinxupquote{data}} (\sphinxstyleliteralemphasis{\sphinxupquote{array like}}\sphinxstyleliteralemphasis{\sphinxupquote{ (}}\sphinxstyleliteralemphasis{\sphinxupquote{sparse vector}}\sphinxstyleliteralemphasis{\sphinxupquote{, }}\sphinxstyleliteralemphasis{\sphinxupquote{same as field type}}\sphinxstyleliteralemphasis{\sphinxupquote{)}}) \textendash{} data each time step

\item {} 
\sphinxstyleliteralstrong{\sphinxupquote{dataMonitor}} (\sphinxstyleliteralemphasis{\sphinxupquote{array like}}\sphinxstyleliteralemphasis{\sphinxupquote{ (}}\sphinxstyleliteralemphasis{\sphinxupquote{sparse vector}}\sphinxstyleliteralemphasis{\sphinxupquote{, }}\sphinxstyleliteralemphasis{\sphinxupquote{same as field type}}\sphinxstyleliteralemphasis{\sphinxupquote{)}}) \textendash{} data container for monitored field

\item {} 
\sphinxstyleliteralstrong{\sphinxupquote{frequency}} (\sphinxstyleliteralemphasis{\sphinxupquote{number}}) \textendash{} frequency to be monitored

\item {} 
\sphinxstyleliteralstrong{\sphinxupquote{lenTimeSignal}} \textendash{} length of timeSignal

\item {} 
\sphinxstyleliteralstrong{\sphinxupquote{stepIndex}} \textendash{} step index within for loop

\item {} 
\sphinxstyleliteralstrong{\sphinxupquote{dtmax}} \textendash{} time step (“sampling”)

\end{itemize}

\item[{Return dataMonitor}] \leavevmode
field monitor result

\item[{Rtype dataMonitor}] \leavevmode
array like (sparse vector, same as field type)

\end{description}\end{quote}

\end{fulllineitems}

\index{fieldProbe() (FIT.FIT method)@\spxentry{fieldProbe()}\spxextra{FIT.FIT method}}

\begin{fulllineitems}
\phantomsection\label{\detokenize{FIT:FIT.FIT.fieldProbe}}\pysiglinewithargsret{\sphinxbfcode{\sphinxupquote{fieldProbe}}}{\emph{data}, \emph{dataMonitor}, \emph{spaceIndex}, \emph{stepIndex}}{}
Field Monitor in time domain for subsequent frequency analysis

dataMonitor container needs to be attribute of a FIT object. Function DOES NOT return anything.
\begin{quote}\begin{description}
\item[{Parameters}] \leavevmode\begin{itemize}
\item {} 
\sphinxstyleliteralstrong{\sphinxupquote{data}} (\sphinxstyleliteralemphasis{\sphinxupquote{array like}}) \textendash{} data each time step

\item {} 
\sphinxstyleliteralstrong{\sphinxupquote{dataMonitor}} (\sphinxstyleliteralemphasis{\sphinxupquote{array like}}) \textendash{} data container for monitored field

\item {} 
\sphinxstyleliteralstrong{\sphinxupquote{spaceIndex}} (\sphinxstyleliteralemphasis{\sphinxupquote{array like}}) \textendash{} FIT canonical index

\item {} 
\sphinxstyleliteralstrong{\sphinxupquote{stepIndex}} (\sphinxstyleliteralemphasis{\sphinxupquote{number}}) \textendash{} iterate time step

\end{itemize}

\end{description}\end{quote}
\begin{description}
\item[{Example canonical index}] \leavevmode
\fvset{hllines={, ,}}%
\begin{sphinxVerbatim}[commandchars=\\\{\}]
\PYG{g+gp}{\PYGZgt{}\PYGZgt{}\PYGZgt{} }\PYG{n}{spaceIndex} \PYG{o}{=} \PYG{n}{np}\PYG{o}{.}\PYG{n}{array}\PYG{p}{(}\PYG{p}{[}\PYG{n}{i} \PYG{k}{for} \PYG{n}{i} \PYG{o+ow}{in} \PYG{n+nb}{range}\PYG{p}{(}\PYG{n}{a}\PYG{o}{.}\PYG{n}{nx}\PYG{o}{*}\PYG{n}{a}\PYG{o}{.}\PYG{n}{ny}\PYG{p}{)}\PYG{p}{]}\PYG{p}{,}\PYG{n}{dtype}\PYG{o}{=}\PYG{n+nb}{int}\PYG{p}{)}\PYG{o}{+}\PYG{l+m+mi}{2}\PYG{o}{*}\PYG{n}{a}\PYG{o}{.}\PYG{n}{np}
\end{sphinxVerbatim}

\end{description}

\end{fulllineitems}

\index{gaussin() (FIT.FIT method)@\spxentry{gaussin()}\spxextra{FIT.FIT method}}

\begin{fulllineitems}
\phantomsection\label{\detokenize{FIT:FIT.FIT.gaussin}}\pysiglinewithargsret{\sphinxbfcode{\sphinxupquote{gaussin}}}{\emph{tdt}, \emph{fmin}, \emph{fmax}}{}
Standard Gaussian-spectrum modulated excitation signal
\begin{quote}\begin{description}
\item[{Parameters}] \leavevmode\begin{itemize}
\item {} 
\sphinxstyleliteralstrong{\sphinxupquote{tdt}} (\sphinxstyleliteralemphasis{\sphinxupquote{array like}}) \textendash{} time signal

\item {} 
\sphinxstyleliteralstrong{\sphinxupquote{fmin}} (\sphinxstyleliteralemphasis{\sphinxupquote{float}}) \textendash{} minimum frequency

\item {} 
\sphinxstyleliteralstrong{\sphinxupquote{fmax}} (\sphinxstyleliteralemphasis{\sphinxupquote{float}}) \textendash{} maximum frequency

\end{itemize}

\item[{Return sig}] \leavevmode
gaussin modulated signal

\item[{Return type}] \leavevmode
array like

\item[{Return npuls}] \leavevmode
normed puls end without trailing zeros

\item[{Rtype npuls}] \leavevmode
array like

\end{description}\end{quote}

\end{fulllineitems}

\index{geomMatrices() (FIT.FIT method)@\spxentry{geomMatrices()}\spxextra{FIT.FIT method}}

\begin{fulllineitems}
\phantomsection\label{\detokenize{FIT:FIT.FIT.geomMatrices}}\pysiglinewithargsret{\sphinxbfcode{\sphinxupquote{geomMatrices}}}{}{}
Geometrical data for FIT to build material matrices

It initializes the attributes:
\begin{quote}\begin{description}
\item[{Variables}] \leavevmode\begin{itemize}
\item {} 
\sphinxstyleliteralstrong{\sphinxupquote{ds}} \textendash{} primary edges

\item {} 
\sphinxstyleliteralstrong{\sphinxupquote{da}} \textendash{} primary facets

\item {} 
\sphinxstyleliteralstrong{\sphinxupquote{dsd}} \textendash{} dual edges

\item {} 
\sphinxstyleliteralstrong{\sphinxupquote{dad}} \textendash{} dual facets

\end{itemize}

\end{description}\end{quote}

\end{fulllineitems}

\index{geomMatricesBOR() (FIT.FIT method)@\spxentry{geomMatricesBOR()}\spxextra{FIT.FIT method}}

\begin{fulllineitems}
\phantomsection\label{\detokenize{FIT:FIT.FIT.geomMatricesBOR}}\pysiglinewithargsret{\sphinxbfcode{\sphinxupquote{geomMatricesBOR}}}{}{}
Geometrical data for FIT: rz code

\begin{sphinxadmonition}{warning}{Warning:}
If called, will raise NotImplementedError
\end{sphinxadmonition}

\end{fulllineitems}

\index{makeMesh() (FIT.FIT method)@\spxentry{makeMesh()}\spxextra{FIT.FIT method}}

\begin{fulllineitems}
\phantomsection\label{\detokenize{FIT:FIT.FIT.makeMesh}}\pysiglinewithargsret{\sphinxbfcode{\sphinxupquote{makeMesh}}}{}{}
Generates meshgrid for cartesian grid

Sets the object’s attributes:
\begin{quote}\begin{description}
\item[{Variables}] \leavevmode\begin{itemize}
\item {} 
\sphinxstyleliteralstrong{\sphinxupquote{xmesh}} \textendash{} xmesh as linspace spanning L\_Min-L\_Max in n points

\item {} 
\sphinxstyleliteralstrong{\sphinxupquote{ymesh}} \textendash{} ymesh as linspace spanning L\_Min-L\_Max in n points

\item {} 
\sphinxstyleliteralstrong{\sphinxupquote{zmesh}} \textendash{} zmesh as linspace spanning L\_Min-L\_Max in n points

\end{itemize}

\end{description}\end{quote}

\end{fulllineitems}

\index{makeMeshBOR() (FIT.FIT method)@\spxentry{makeMeshBOR()}\spxextra{FIT.FIT method}}

\begin{fulllineitems}
\phantomsection\label{\detokenize{FIT:FIT.FIT.makeMeshBOR}}\pysiglinewithargsret{\sphinxbfcode{\sphinxupquote{makeMeshBOR}}}{}{}
Generates meshgrid for rz grid

\begin{sphinxadmonition}{warning}{Warning:}
If called, will raise NotImplementedError
\end{sphinxadmonition}

\end{fulllineitems}

\index{nulldiv() (FIT.FIT method)@\spxentry{nulldiv()}\spxextra{FIT.FIT method}}

\begin{fulllineitems}
\phantomsection\label{\detokenize{FIT:FIT.FIT.nulldiv}}\pysiglinewithargsret{\sphinxbfcode{\sphinxupquote{nulldiv}}}{\emph{a}, \emph{b}}{}
Calculates a/b elementwise if b contains 0

Division doing multiplication with nullinv essentially

can handle some python sparse data, attribute a.shape is a necessity
\begin{quote}\begin{description}
\item[{Parameters}] \leavevmode\begin{itemize}
\item {} 
\sphinxstyleliteralstrong{\sphinxupquote{a}} (\sphinxstyleliteralemphasis{\sphinxupquote{array like}}) \textendash{} numerator

\item {} 
\sphinxstyleliteralstrong{\sphinxupquote{b}} (\sphinxstyleliteralemphasis{\sphinxupquote{array like}}) \textendash{} denominator

\end{itemize}

\item[{Returns}] \leavevmode
pseudo divison

\item[{Return type}] \leavevmode
array like

\end{description}\end{quote}

\end{fulllineitems}

\index{nullinv() (FIT.FIT method)@\spxentry{nullinv()}\spxextra{FIT.FIT method}}

\begin{fulllineitems}
\phantomsection\label{\detokenize{FIT:FIT.FIT.nullinv}}\pysiglinewithargsret{\sphinxbfcode{\sphinxupquote{nullinv}}}{\emph{a}}{}
Matrix (vector) inversion by replacing diagonal entries with its reciproke value v = 1/v

can handle some python sparse data, attribute a.shape is a necessity
\begin{quote}\begin{description}
\item[{Parameters}] \leavevmode
\sphinxstyleliteralstrong{\sphinxupquote{a}} (\sphinxstyleliteralemphasis{\sphinxupquote{array like}}) \textendash{} input to be inverted

\item[{Returns}] \leavevmode
pseudo inverse

\item[{Return type}] \leavevmode
array like

\end{description}\end{quote}

\end{fulllineitems}

\index{plotField() (FIT.FIT method)@\spxentry{plotField()}\spxextra{FIT.FIT method}}

\begin{fulllineitems}
\phantomsection\label{\detokenize{FIT:FIT.FIT.plotField}}\pysiglinewithargsret{\sphinxbfcode{\sphinxupquote{plotField}}}{\emph{X}, \emph{Y}, \emph{field}, \emph{figureNumber}, \emph{title}}{}
This function produces a 3D plot for a field quantity in one cutting plane X,Y
\begin{quote}\begin{description}
\item[{Parameters}] \leavevmode\begin{itemize}
\item {} 
\sphinxstyleliteralstrong{\sphinxupquote{X}}\sphinxstyleliteralstrong{\sphinxupquote{,}}\sphinxstyleliteralstrong{\sphinxupquote{Y}} (\sphinxstyleliteralemphasis{\sphinxupquote{array like}}) \textendash{} meshgrid representing a 2D cutting plane

\item {} 
\sphinxstyleliteralstrong{\sphinxupquote{field}} (\sphinxstyleliteralemphasis{\sphinxupquote{array like}}\sphinxstyleliteralemphasis{\sphinxupquote{ (}}\sphinxstyleliteralemphasis{\sphinxupquote{dense matrix}}\sphinxstyleliteralemphasis{\sphinxupquote{)}}) \textendash{} field quantity scalar representation fitting mesh X,Y

\item {} 
\sphinxstyleliteralstrong{\sphinxupquote{title}} (\sphinxstyleliteralemphasis{\sphinxupquote{string}}) \textendash{} title of the plot figure

\end{itemize}

\item[{FigureNumber}] \leavevmode
convenience to assigne dedicated plot window

\item[{Returns}] \leavevmode
axis handler of python plot axis3D object

\item[{Return type}] \leavevmode
matplotlib.axes.\_subplots.Axes3DSubplot

\end{description}\end{quote}

\end{fulllineitems}

\index{selectiveDFT() (FIT.FIT method)@\spxentry{selectiveDFT()}\spxextra{FIT.FIT method}}

\begin{fulllineitems}
\phantomsection\label{\detokenize{FIT:FIT.FIT.selectiveDFT}}\pysiglinewithargsret{\sphinxbfcode{\sphinxupquote{selectiveDFT}}}{\emph{dftMatrix}, \emph{timeSignal}}{}
A specalized version of the discrete fourier transform, giving only frequency selective result for a given sampling vector in frequency domain. It is well suited for short time signals, that therefore have poor frequency resolution using a normal FFT, for long time signal the effort is excessive. It uses a linear operator, the DFT matrix, to optimize the calculation speed. This was made for UQ implementations where a frequency curve needs to be calculated for each of several runs with a time signal of equal length. Cutting the linear operator allows DFT of parts of the time signal.

\begin{sphinxadmonition}{warning}{Warning:}
Pay attention to sampling. Time Signal needs to be long enough to give the correct frequency representation t\_max \textgreater{} 1/f\_min
\end{sphinxadmonition}
\begin{quote}\begin{description}
\item[{Parameters}] \leavevmode\begin{itemize}
\item {} 
\sphinxstyleliteralstrong{\sphinxupquote{dftMatrix}} (\sphinxstyleliteralemphasis{\sphinxupquote{numpy matrix}}) \textendash{} DFT matrix generated for discrete, selective frequency points

\item {} 
\sphinxstyleliteralstrong{\sphinxupquote{timeSignal}} (\sphinxstyleliteralemphasis{\sphinxupquote{numpy array}}) \textendash{} time signal generated with \sphinxtitleref{fieldMonitorTD()}

\end{itemize}

\item[{Returns}] \leavevmode
frequency representation of timeSignal

\item[{Return type}] \leavevmode
numpy matrix

\end{description}\end{quote}

\end{fulllineitems}

\index{setGeometry() (FIT.FIT method)@\spxentry{setGeometry()}\spxextra{FIT.FIT method}}

\begin{fulllineitems}
\phantomsection\label{\detokenize{FIT:FIT.FIT.setGeometry}}\pysiglinewithargsret{\sphinxbfcode{\sphinxupquote{setGeometry}}}{\emph{nx}, \emph{ny}, \emph{nz}, \emph{LxMin}, \emph{LyMin}, \emph{LzMin}, \emph{LxMax}, \emph{LyMax}, \emph{LzMax}}{}
This functions sets the basic geometrical shape representation.
\begin{quote}\begin{description}
\item[{Parameters}] \leavevmode\begin{itemize}
\item {} 
\sphinxstyleliteralstrong{\sphinxupquote{N}} (\sphinxstyleliteralemphasis{\sphinxupquote{integer}}) \textendash{} number of points in \_

\item {} 
\sphinxstyleliteralstrong{\sphinxupquote{L\_Min}} (\sphinxstyleliteralemphasis{\sphinxupquote{float}}) \textendash{} total length in \_ (minimum)

\item {} 
\sphinxstyleliteralstrong{\sphinxupquote{L\_Max}} (\sphinxstyleliteralemphasis{\sphinxupquote{float}}) \textendash{} total length in \_ (maximum)

\end{itemize}

\end{description}\end{quote}

All of the above are then accessible via attribute calling convention
\begin{description}
\item[{Example usage}] \leavevmode
\fvset{hllines={, ,}}%
\begin{sphinxVerbatim}[commandchars=\\\{\}]
\PYG{g+gp}{\PYGZgt{}\PYGZgt{}\PYGZgt{} }\PYG{n}{fitObject}\PYG{o}{.}\PYG{n}{nx}
\end{sphinxVerbatim}

\end{description}

\end{fulllineitems}


\end{fulllineitems}

\index{main() (in module FIT)@\spxentry{main()}\spxextra{in module FIT}}

\begin{fulllineitems}
\phantomsection\label{\detokenize{FIT:FIT.main}}\pysiglinewithargsret{\sphinxcode{\sphinxupquote{FIT.}}\sphinxbfcode{\sphinxupquote{main}}}{}{}
Example to calculate steady state in rectangular box resonator

main frame protected execution (means: only runs if directly executed. Does not run at import modul statements)

\end{fulllineitems}



\chapter{Indices and tables}
\label{\detokenize{index:indices-and-tables}}\begin{itemize}
\item {} 
\DUrole{xref,std,std-ref}{genindex}

\item {} 
\DUrole{xref,std,std-ref}{modindex}

\item {} 
\DUrole{xref,std,std-ref}{search}

\end{itemize}


\renewcommand{\indexname}{Python Module Index}
\begin{sphinxtheindex}
\let\bigletter\sphinxstyleindexlettergroup
\bigletter{f}
\item\relax\sphinxstyleindexentry{FIT}\sphinxstyleindexpageref{FIT:\detokenize{module-FIT}}
\end{sphinxtheindex}

\renewcommand{\indexname}{Index}
\printindex
\end{document}